% Created 2022-02-13 Sun 15:40
% Intended LaTeX compiler: pdflatex
\documentclass[11pt]{article}
\usepackage[utf8]{inputenc}
\usepackage[T1]{fontenc}
\usepackage{graphicx}
\usepackage{longtable}
\usepackage{wrapfig}
\usepackage{rotating}
\usepackage[normalem]{ulem}
\usepackage{amsmath}
\usepackage{amssymb}
\usepackage{capt-of}
\usepackage{hyperref}
\author{Anak Wannaphaschaiyong}
\date{\today}
\title{Org Mode Note}
\hypersetup{
 pdfauthor={Anak Wannaphaschaiyong},
 pdftitle={Org Mode Note},
 pdfkeywords={},
 pdfsubject={},
 pdfcreator={Emacs 27.1 (Org mode 9.6)}, 
 pdflang={English}}
\begin{document}

\maketitle
\tableofcontents


\section{types of header parameters}
\label{sec:orgff005fd}
According to \href{file:///home/awannaphasch2016/org/notes/emacs/packages/org-babel.org}{Introduction to literate programming with org-mode and org-babel by howard}

\begin{quote}
With the basics in place, the rest of this tutorial describes the source block controls done by parameter settings. I’ve separated and organized these based on its usage:

Evaluation Parameters
like dir, affects how a block is executed
Export Parameters
affects how a block or the results from execution is shown when it is exported to HTML
Literate Programming Parameters
connecting blocks together to change the actual source code
Variable Parameters
variables for a source block can be set in various ways
Miscellaneous Input/Output
of course, you have to have a collection of parameters that don’t fit elsewhere
\end{quote}

\section{Key binding}
\label{sec:orgd5282eb}
\begin{center}
\begin{tabular}{lllll}
commands & key & doom key & custom key & description\\
org-footnote-action &  & spc-m-f &  & create footnotes\\
org-set-property & c-c c-x p & spc-m-o &  & set properties of file\\
\end{tabular}
\end{center}
\end{document}
